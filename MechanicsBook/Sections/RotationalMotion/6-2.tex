\begin{tikzpicture}
\draw (0,0) circle (0.0001cm);
\filldraw (6,0) circle (0.05cm);
\draw (6, 0) -- (8, 1) -- (7, 3) -- (5, 2) -- (6,0);
\end{tikzpicture}

Now we are going to discuss perhaps the most boring topic that is covered in AP Physics C. That is, of course, constant angular acceleration problems. These problems are nearly identical to the linear constant acceleration problems so we will discuss these problems in brief and then talk about how we can relate the angular motion of the object and the linear motion of points on that object. First, we have the simplest relation which is that $\omega_0+\alpha t=\omega$, or equivalently, $\Delta \omega=\alpha t$. We also have that \begin{equation}\Delta \theta=\omega_0 t+\alpha \frac{t^2}{2}\end{equation} Or equivalently, \begin{equation}\Delta \theta=\omega_{final}t-\alpha \frac{t^2}{2}\end{equation} Additionally, we have that \begin{equation}\omega_{final}^2 = \omega_{0}^2+2\alpha \Delta \theta\end{equation} This is essentially the exact same formula as what we had with linear acceleration and we will be able to solve the problems the same way. 

It is necessary to discuss how we can relate linear and angular motion. We will first discuss this idea using the idea of a point on the edge of a frisbee. We can think about how much the point on the edge of the frisbee moves during one second to find the linear velocity of the point. The formula we find will be surprisingly simple. So if the frisbee rotates with a constant angular velocity $\omega$, then in $\frac{2\pi}{\omega}$ seconds, the object will have completed a complete revolution. This is because $$\Delta \theta=\omega \Delta t$$ If we plug in, we find that $$\Delta \theta=\omega \frac{radians}{s} \cdot \frac{2\pi}{\omega}s=2\pi \ radians$$ which corresponds to a complete revolution. So the period of the rotation of the frisbee is $\frac{2\pi}{\omega} \ seconds$. Now, we need to think about how much distance the point on the frisbee will travel during one rotation. We can assume the frisbee is circular so, during one rotation, the point should travel $2\pi r$ meters. Where $r$ is the radius of the frisbee. So if the object travels $2\pi r$ meters in $\frac{2\pi}{\omega}$ seconds, then the linear velocity of the object is $\omega r$ seconds. So we say that the linear velocity of the point is $v=\omega r$. In this case, $\omega$ is the angular velocity of the frisbee, and $r$ is the distance from the center of rotation to the point. This result hold for other objects that are not necessarily frisbees and for other points on the solid that are not necessarily on the edge of the surface. A good example might be a rectangle that is rotating about one of its corners with an angular velocity of $\omega r$ in the plane. This means that every $\frac{2\pi}{\omega}r$ seconds, the object will be in the same place that it was before and will have completed a complete rotation. We can say that the length of the rectangle is $l$ and the width is $w$. The angular velocity of the corner opposite to the one that is being rotated about is $\omega_r r$ where $r$ is the distance from the corner being rotated about to the opposite corner. Using the Pythagorean theorem, it is simple to see that $$r=\sqrt{l^2+w^2}$$ So the linear velocity of this point is $$\omega \sqrt{l^2+w^2}$$ Now, I only showed why in the case of constant angular velocity, even if the angular velocity is changing, $\omega_r r$ will still be equal to $v$. So if we have $\omega_r r=v$, we can differentiate both sides with respect to time and the two sides should still be equal because of $\omega_r r=v$ for all possible angular and linear velocities. However, if $\omega_r r=v$ were only true for this angular velocity, we would not be able to do this. So $\frac{dv}{dt}=a$ and $r$ is constant because the distance from the point we are recording to the center of rotation should not be changing with time. So $\omega_r r$ just equals $$r\frac{d\omega_r}{dt}=\alpha r$$ So the linear acceleration of a moving point on a rotating rigid body is the angular acceleration times the distance from the point to the center of rotation. This can, of course, be continued for further time derivatives but these will not be encountered, so we do not need to worry about them here. It is tough to have any intuition for these formulas, and they do not seem very applicable, but as we will see them later when discussing the rolling motion. 