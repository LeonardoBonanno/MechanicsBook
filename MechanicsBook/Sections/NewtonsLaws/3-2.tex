\begin{tikzpicture}
\draw[circle] (0,0) circle (0.00001cm);
\draw[->] (3,0) -- (9,4);
\draw[dashed] (3,0) -- (9,0);
\draw[dashed] (9,0) -- (9,4);
\draw[black] (6,-0.5) node[anchor=west] {$F_x$};
\draw[black] (9,1.75) node[anchor=west] {$F_y$};
\draw[black] (5.5,2.5) node[anchor=west] {\begin{turn}{30} $\vec{F}$ \end{turn}};
\end{tikzpicture}
\newline
When we want to start talking about Newtonian dynamics, we first want to define what exactly this term means. We can think of dynamics as being what happens when we have a set of objects imparting certain forces on each other. This is different from kinematics where we are given information about the motion of the objects, and our goal is to describe this motion more thoroughly. I am sure you have heard of Newton’s three laws, and they are very relevant in this discussion. We will now discuss the first one. The first law says that a body undergoing constant velocity motion(which could include being at rest), will stay in this state unless acted upon by force. This means that an object could be going extremely fast, and if nothing goes to slow it down, it will continue at that pace. When I first heard of this law, I found it extremely counter-intuitive. The first reason was that when I was younger, I had always heard that bodies would try to slow down. This was incorrect. As weird as it may seem, it is well tested and documented that objects will not change their speed unless acted upon by an external force. Let us think about some of the consequences this. The first idea is that if we were to have anybody moving at a constant speed $v$ the net force acting on the object must be 0. We can also think about the ramifications if a force is zero in one direction but non-zero in other directions. For now, all we know is that in the direction in which the force is zero, the velocity in that direction must stay constant. We can apply this to free fall. A body undergoing free fall has a force acting on it in the y-direction, but not in the x-direction, so its x-velocity stays the same. 

After discussing the rather qualitative first law, we can now garner a quantitative understanding of the connection between forces and changes in velocity with Newton’s second law. Newton’s second law is deceptively simple. Often we will get problems where we are told to use Newton’s laws. However, often, the insight that allows us to solve the problem will not come from Newton. Newton’s Second law states that the force imparted on an object is equal to its change in momentum over a change in time. For now, we can think of momentum as being the mass(which we will assume to be constant) of a body times the velocity of the body, which we can assume will be changing. So what is the change in mass times velocity divided by a change in time? Well, this is just the mass times the acceleration. From here we get the famous equation. \begin{equation}\vec{F}=m\vec{a}\end{equation} First, we note that acceleration is a vector, so the force is a vector as well. The relation holds in the $x,y$ and $z$ directions and, fundamentally in every direction. We will use this fact later. Another thing to note is how we can derive the first law from the second law. If we set the acceleration equal to 0, we will see that $F=0$. This relation works the other way. If the force is 0, the acceleration must be as well. This is important to conceptualize because questions involving this idea are brought up very frequently. Another important thing to remember is that \begin{equation}\vec{F_{net}}=m\vec{a}\end{equation} $F_{net}$ represents the vector sum of all of the forces acting on an object. So, if there is a force of $F_1$ acting in the I direction on an object and a force of $F_1$ acting in the -I direction, the body will experience no force in the I direction. These are fundamental concepts, and if you do not understand them, I would recommend for you to reread this section. Lastly: Please note that the units of a force are the units of the mass times the units of acceleration. Normally we take this as being $kg \ \frac{m}{s^2}$. However, because physicists like to be concise, this is almost always written using a $N$, for newtons. 