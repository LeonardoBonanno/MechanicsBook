A large number of elementary physics problems include problems with pulleys and with ideas surrounding tension forces. Frankly, I think that these are some of the easier problems on the AP Physics exam and in elementary physics in general. Let us examine the classic problem of two blocks of mass m and M on opposite ends of a massless string being overlaid on a massless pulley in Fig. 3.6.1. These are strong assumptions, but they are made frequently. For now, we have to assume these facts. We can assume that the pulley is being held up by something and is not going to move. Now we want to think about what is going to happen here. Physically, your intuition should be pretty clear. The larger mass should fall down and while doing so pull up the smaller mass. If this was your intuition, very good! If not, I would recommend thinking about this for a few minutes before we get to the equations. 
\newline
\newline
\begin{tikzpicture}
\draw[black] (0,0) circle (0.001cm);
\draw[black] (6,0) circle (1cm);
\draw[black, ->] (8,0.25) -- (8,-3);
\draw[black, ->] (4,-3) -- (4,0.25);
\draw[black] (5,0) -- (5,-3);
\draw[black, ->] (7,-2);
\draw[black, ->] (5,-2);
\draw[black] (7,0) -- (7,-3);
\draw[black] (6.75,-3.75) node[anchor=west] {$M$};
\draw[black] (4.75,-3.5) node[anchor=west] {$m$};
\draw[black] (5.5,-3) rectangle (4.5,-4);
\draw[black] (7.75,-3) rectangle (6.25,-4.5);
\end{tikzpicture}
\begin{center} 
(Fig. 3.6.1)
\end{center}
Armed with this intuition, We can write the equations of motion for the two masses. We can assume that the force of tension acting on both of the masses will be the same. We will take that the positive direction is in the direction of the pulley and the downward direction will be in the direction of the gravitational force. We can then write that \begin{equation}ma_1 = F_{T}-mg\end{equation} and \begin{equation}Ma_2 = F_T-Mg\end{equation} If you do not see why this is, look at the diagram and draw free-body diagrams. We have these two equations, and we are trying to solve for the accelerations of the two blocks. At this point, you may be thinking that we need to be given the tension force to solve for $a_1$ and $a_2$, but this is not the case. To solve this problem, we need to have a physical insight into the nature of a pulley. How can we relate the accelerations of the two pulleys? Think about what is happening as the blocks are going up and down. What must be the case? If you have thought about this for a few minutes, or if you found it immediately, I encourage you to solve the rest of the problem on your own, if not, I recommend you think about it until you find it. 

The insight is that $a_1=-a_2$. Why must this be the case? Because for every inch the less massive block goes up, the more massive block must go down by an inch. This is simply because the string has a fixed length. So we can write \begin{equation}L_1+L_2=L\end{equation}where $L$ is the length of the string, and $L_1$ and $L_2$ are the distances from the masses to the top and middle of the pulley, market point $A$. $L$ is a constant so we can differentiate both sides with respect to t to find that \begin{equation}\frac{dL_1}{dt}+\frac{dL_2}{dt}=0\end{equation} We can differentiate again and rearrange to find that \begin{equation}\frac{d^2L_1}{dt^2}=\frac{-d^2L_2}{dt^2}\end{equation} But wait, $\frac{d^2L_1}{dt^2}$ is just the acceleration of $m$ and $\frac{d^2L_2}{dt^2}$ is just the mass of $M$. This may seem like a rather silly way to come to this conclusion, but if you did not have the original physical insight, you could have found this fact about the accelerations as we have shown. Now we can go about solving the problem. We write that $$ma_1=F_{Tension}-mg$$ and \begin{equation}-Ma_1=F_{Tension}-Mg\end{equation} So if we subtract that second expression from the first one we find that \begin{equation}a_1\left(m+M\right)=\left(M-m\right)g\end{equation} So we finally find that $$a_1=\frac{\left(M-m\right)g}{\left(m+m\right)}$$ And therefore $$a_2=\frac{g \left (m-M\right)}{\left(m+m\right)}$$ This is a classical result in elementary physics, and I find it quite satisfying. However, we should still check it to see if it matches with our intuition. First, let us look at what happens if $M>m$. Then $a_1$ is positive and mass $m$ moves up while $a_2$ is negative and mass $M$ moves down. This is intuitive. If we have two masses and one is smaller, then the larger one should go down while the smaller one goes up. This train of logic works the same if $M<m$. I encourage you as a personal exercise now to find the tension force that the string imparts on the two objects. Also, I want you to find another way you could have solved the problem. A hint would be that it plays with our fixed notions as up and down as necessarily being distinct. This way of solving the problem allows the direction of a curve to be positive and the opposite direction of the curve to be negative. This way of solving the problem is quite a lot easier, and I think you will gain quite a bit from thinking about it on your own. 
\clearpage