\begin{tikzpicture}
\draw (0,0) -- (12,0);
\draw (3,0) -- (3,1) -- (9,1) -- (9,0);
\draw (3.25,1) -- (3.75,1.5);
\draw (4.25,1) -- (3.75,1.5);
\draw (3.75,1.5) -- (3.75,2.5);
\draw[->] (3.875,2) -- (4.5,2);
\draw (3.75,2.85) circle (0.35cm);
\end{tikzpicture}
\newline
The center of mass is an exquisite way to describe complex systems. Intuitively, the center of mass is the place where we can think all of the mass of the system as being located. Of course, this is not the case, but we can perform many calculations using this fact. The center of mass is a weighted average of the positions of the objects in a system with the weight being calculated with respect to the masses of the objects. Now we will see that we can think about conservation of momentum in terms of the conservation of momentum for the center of mass of a system. First, let us define the center of mass in one direction. \begin{equation}X_{CM}= \frac{\sum x_i m_i}{\sum m_i}\end{equation} The $x_i$’s here represent the positions of the masses in the systems, and the $m_i$’s are just the masses of the objects in the system. They can be negative depending on where we decide to make the position zero. Essentially, we sum the position of the object times its mass for each mass, and once we find that total sum, we divide the whole thing by the sum of all of the masses. So we can see that we have to multiply the position of the object times the mass of the object which gives more “weight” to the objects with greater mass. I was confused when I first encountered definition because it is not clear what the $x_i$ are. If we have two masses, who is to say which one is where? Well, we do not have to worry about this because the answer we will get will be the same no matter where we start. So, for example, let us imagine a system with two masses $m_1$ and $m_2$ separated by a distance $d$, and we are asked to find the center of mass of the system. This is quite an easy task. We just assume that $m_1$ or $m_2$ is at a position we call zero. Alternatively, we could assume that $m_2$ is at zero, or that the point directly in between them is zero. Ultimately, we will get the same answer. This is because the center of the mass will be relative to the point that we choose as 0. Let us do an example to clarify what I mean by this. So let us first say that $m_1$ is at zero. So the position of $m_1$ is zero and the position of $m_2$ is $d$, because $m_2$ is $d$ away from 0. We can therefore write that $$x_{cm} = \frac{m_2d+ m_10}{m_1+m+_2}$$ We simplify this to find that $$x_{cm} =\frac{m_2 d}{m_1+m_2}$$ this represents the center of mass as being $\frac{m_2 d}{m_1+m_2}$ away from $m_1$ in the direction of $m_2$. Now let us assume that $m_2$ is our zero point. $$x=\frac{0\cdot m_2+d\cdot m_1}{m_1+m_2}$$ You may now be thinking that we have gotten a different answer, but this is not, in fact, the case. This answer says that the center of mass is $\frac{dm_1}{m_1+m_2}$ away from $m_2$ in the direction of $m_1$. I challenge you to find the reason why these are the same answer. All it takes is some simple algebra and addition. Now, if you are still not convinced let us start from the point $\frac{d}{2}$ between the two objects exactly and that the x-axis goes from $m_1$ to $m_2$.  In this case \begin{equation}x=\frac{\frac{-d}{2}m_1+\frac{d}{2}m_2}{m_1+m_2}=d\frac{m_2-m_1}{m_1+m_2}\end{equation} Note that we have said the position of $m_1$ is negative because it is to the left of the middle between the two points and is therefore negative. This is once again the same answer. I challenge you to convince yourself of it using the same method you used above. There are some crazy things we can do using the center of mass. But first, we need to understand that the momentum of the center of mass will not change unless a force is acting upon the system. This makes sense because we can represent a whole system as just being the mass of the system at the point location of the center of mass. From this standpoint, we are just looking at how a mass will not change its momentum unless acted upon by an outside force. 

Let us do an example to make this more clear. We have a person on one side a box sliding along an ice court(assume an ice court is a frictionless plane). Our problem is to look at how much the box will move if the person will move to the other side of the box and there are no external forces. We assume the person has a mass $m_p$, the box has a mass $m_b$, and the length of the box is $l$. We can assume the person is so thin that they are exactly at the edge of the box. Initially, taking the middle of the boat to be the 0 point, the center of mass is \begin{equation}_{CM_1} = \frac{lm_p}{2\left(m_p+m_b\right)}\end{equation} After the person moves to the other side of the box, we can also calculate the center of mass as \begin{equation}X_{CM_2} = \frac{-lm_p}{2\left(m_p+m_b\right)}\end{equation} This result comes from taking the new location of the middle of the box as being 0. No external force was acting on the system so $$X_{CM_1}=X_{CM_2}$$ Therefore, for the center of mass to not change, the box must have moved $$\frac{lm_p}{m_p+m_b}$$ to the right.

Now that we have covered the example of the box on the frictionless plane, we should talk about another classic example: an astronaut floating away from their station in space with a tool in their hand. First, I want you to think for yourself about what happens when you throw something away from yourself back down on earth. Unless you are throwing something very quickly, nothing happens. This is because you and the ball are not an isolated system on earth. You are still being acted upon by a frictional force that acts against your motion. However, the frictional force is not there for an astronaut. So, using the same logic about the conservation of mass that we used for the boat example, we can reason about the motion of the astronaut. If the astronaut-tool system initially had no momentum and therefore a fixed center of mass, this will not change when the tool is thrown. So when the tool is thrown in one direction, the astronaut will go the other direction. This is quite useful for the astronaut, who would be floating away if he/she had not had the tool. 

The last thing we need to talk about is the fact that the center of mass can also be used if the system has an infinite number of masses. In this case, we can not rely on a simple formula, so we need to derive one that uses integrals to account for this type of situation. Fundamentally, the center of mass is the average location of mass within a system. That is all our formula was. Think about the ramifications of this and justify it to yourself with some simple examples. From calculus, we know that the average of a function over an interval is $$\frac{1}{length \ of \ interval}\int{f\left(x\right) dx}$$ For us, the interval we are looking at is all of the masses in the system. So the “length of the interval,” is just the sum of all of the masses in the system. Also, the $f\left(x\right)$ that we are averaging for is the position of each mass, and $dx$(represent each piece of the interval), is each piece of mass for us. So, if we write position as $r$, we get that \begin{equation}x=\frac{1}{M}\int{rdm}\end{equation} This formula may seem odd because we have never integrated over a mass before and we almost never do this integral directly. Normally, we will be given an expression for the mass density such as $\lambda=kr$. Where $k$ is a constant, $\lambda$ is the density of a small length of mass, and $r$ tells us how the density varies with a position at. I encourage you to do the example I have set up above with $\lambda=kr$ over a very thin stick(so thick that the stick is essentially a line). Imagine that the density at one end of the stick is 0 and at the other is $kl$, where $l$ is the length of the stick. This is a confusing topic, so do not worry if you get stuck. You only need to understand that $$dm=\lambda dr$$ and that ultimately, we are finding the average location of mass in the system. 

\pagebreak