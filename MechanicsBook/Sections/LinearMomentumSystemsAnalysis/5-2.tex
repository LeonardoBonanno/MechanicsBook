\begin{tikzpicture}
\draw[black] (0,0) circle (0.0001cm);
\draw[black] (3,0) circle (0.5cm);
\draw[black] (9,0) circle (0.5cm);
\draw[dashed] (3,-0.5) -- (5.75,-0.5) -- (5.75,-3);
\draw[dashed] (9,-0.5) -- (6.25,-0.5) -- (6.25,-3);
\draw (4.3,-0.25)node[anchor=west] {$V_{1x}$};
\draw (7.3,-0.25)node[anchor=west] {$V_{2x}$};
\draw (5,-1.5) node[anchor=west] {$V_{1y}$};
\draw (6.25,-1.5) node[anchor=west] {$V_{2y}$};
\draw (2.625,0) node[anchor=west] {$M_1$};
\draw (8.625,0) node[anchor=west] {$M_2$};
\draw[->] (3,-0.5) -- (5.75,-3);
\draw[black] (5.5,-3.25) node[anchor=west] {$v_1$};
\draw[black] (6,-3.25) node[anchor=west] {$v_2$};
\draw[->] (9,-0.5) -- (6.25,-3);
\end{tikzpicture}
\begin{center}
(Figure 5.2.1)
\end{center}
\
Now that we have introduced momentum we will describe collisions between two objects. Often these objects are described as square boxes but to make it more interesting and realistic; let us think of two gas particles colliding in the air around us. We will also assume that the two particles are so far away from any other particles, that they make up their isolated system. A few things can happen during collisions between particles, and they are all intuitive. The objects can collide and stay together, or they can collide and go separate ways. In the real world, we might additionally see a transfer of mass between the two particles but here, and for all AP Physics purposes, we will assume that this does not happen. So, we will now introduce the law of conservation of momentum. This law is pretty simple and is self-describing. The law says that the sum of the momentum of the particles before the collision is equal to the sum of the momentum of the particles after the collision,  assuming the system is isolated. This is identical to saying that the center of mass of the system has the same momentum before and after a collision. This law is not necessarily intuitive; in fact, I found it very hard to understand what this law was saying when I first learned about it. However, I feel that doing a few examples applying the concept can help build your intuition for it. First, let us assume that two objects of equal mass are moving directly towards each other with the same velocity. Let us assume that the objects are sticky and attach right after they collide. We assume that the collision happens extremely rapidly, so there is not an intermediate period during which the objects are not entirely connected. Here, we can essentially assume that the collision period is arbitrarily small. So what happens? Well, the objects will attach and switch the directions of their velocity but keep the magnitude of their velocity the same. Let us verify this using conservation of momentum.  So if we assign the axis on which the bodies are moving as the x-axis, the velocities of the bodies can be written as $v$ and $-v$. So if the bodies have the same mass, then the momentums of the bodies are $m\vec{v}$ and $-m\vec{v}$. So the total momentum in the x-direction before the collision was zero, and therefore the total collision after the collision is also. But if the two objects are attached, and the momentum of the system is zero, we can write \begin{equation}0=\vec{v}_{final} \left(m+M\right)\end{equation} Here we are treating the two-mass system as if it was a single mass but this is completely fine because they are attached, so they are essentially a single mass. From Eqn. 5.5.2, it should be fairly clear to see why the velocity of the two masses of the system after the collision was zero. Now let us do another example where the two objects stick together but where the colliding objects do not have the same masses and velocities. 

Let us assume that two objects of mass $m_1$ and $m_2$ are sliding in a frictionless plane with velocities $\vec{v}_1$ and $\vec{v}_2$ when they collide and attach. The momentum of the system before the collision was $m_1\vec{v}_1+m_2\vec{v}_2$ and it should stay that way until after the collision as well. But, the momentum of the system after the collision is \begin{equation}\vec{v}_{final}\left(m_1+m_2 \right)\end{equation} When we set these two expressions equal to each other we find that \begin{equation}\vec{v}_{final}=\frac{m_1\vec{v}_1+m_2\vec{v}_2}{\left(m_1+m_2\right)}\end{equation} This is a very elegant result, and we can look at some limiting cases to test if our expression is correct. First, let us assume that $m_1$ and $v_1$ are much greater than $m_2$ and $v_2$ respectively. This is essentially the scenario we have when a large bowling ball is coming by and hitting a ping pong ball is at rest. Let us assume somehow in our scenario that the hole of the bowling ball takes in the ping pong ball. Well in this case we have that  $$\vec{v_{final}} \approx \frac{m_1 v_1}{M_1}$$ Because $m_2\vec{v}_2$ and $m_2$ are negligible compared to $m_1\vec{v_1}$ and $m_1$ respectively. $\vec{v}_{final}$ is just $\vec{v}_1$ when simplifying. This should be intuitive. When a large object is coming faster hits and attaches itself to a tiny object at rest, the velocity essentially stays the same. What you may have noticed in these examples is that the kinetic energy of the objects does not stay constant. The kinetic energy has decreased in all of the examples we have seen. This may be especially disconcerting knowing that we have assumed there is no friction. Well, we have not violated the law of conservation of energy. Instead, the kinetic energy of the objects is lost to heat, sound, deformations, and many other things when the two objects collide. 

Another concern you may be having now surrounds objects under free fall. It appears as if the velocity of the ball is just rapidly increasing, which seems to be violating the law that momentum must be conserved. However, we have failed to take the whole system into account. In our model, we have looked only at the speed of the object changes. Indeed, if the ball were increasing in speed on its own, this would be impossible. However, the object does not make up its isolated system. Instead, the object is part of the earth-object system. So we have to take into account both the motion of the earth and the object to get the full picture of momentum. It turns out, when we do the calculations in later sections, that the sum of the momentum of the earth and the object during free fall stays constant. This may seem remarkable, but remember, this is just how nature is. 

When a collision occurs, and kinetic energy is conserved, we have what is called an elastic collision. When a collision occurs, and kinetic energy is not conserved, we have an inelastic collision. When we have a collision where the two bodies collide and stick together, this is a perfectly inelastic collision. It is called by this name because it entails the largest loss of kinetic energy possible while still obeying the conservation of momentum. We are now going to derive what happens in an elastic collision. In an elastic collision we have \begin{equation}m_1 v_1+m_2 v_2=m_1v_{a1}+m_2v_{a2}\end{equation} Where $\vec{v}_{1a}$ and $\vec{v}_{2a}$ are the velocities of $m_1$ and $m_2$ respectively after the collision. We have also assumed conservation of energy so \begin{equation}m_1\left(v_1\right)^2+m_2\left(v_2\right)^2=m_1\left(v_{a1}\right)^2+m_2\left(v_{a2}\right)^2\end{equation} I have gotten rid of the vectors here because when dealing with kinetic energy, we only have to think about the magnitudes of the velocities. I have also gotten rid of the factor of $\frac{1}{2}$ in front of each term because they cancel out. We will assume that $m_1$, $m_2$, $v_1$, and $v_2$ are known. So we have two equations in which there are two unknowns, $v_{a1}$ and $v_{a2}$. So, we should be able to solve this by substitution and using the quadratic formula. I will let you deal with these equations on their own as these formulas are used frequently, and useful insight can come from understanding them other than testing them at asymptotic cases. 