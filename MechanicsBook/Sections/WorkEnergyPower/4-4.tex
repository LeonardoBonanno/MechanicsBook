\begin{tikzpicture}
\draw[black] (0,0) circle (0.0001cm);
\draw[black] (3,1) rectangle (9,6);
\draw[black] (4.125,5) node[anchor=west] {$\oiint_S \vec{E} \cdot d\vec{A} = \frac{Q}{\epsilon_0}$};
\draw[black] (4.125,4) node[anchor=west] {$\oiint_S \vec{B} \cdot d\vec{A} = 0$};'\draw[black] (4.125,3) node[anchor=west] {$ \oint_P \vec{B} \cdot d\vec{r} = \mu_o I + \epsilon_o \mu_o \frac{\partial \Phi_E}{\partial t}$};
\draw[black] (4.125,2) node[anchor=west] {$ \oint_P \vec{E} \cdot d\vec{r} = \frac{-\partial \Phi_B}{\partial t}$};
\end{tikzpicture}
\begin{center}
(Figure 4.4.1)
\end{center}
Above are the Maxwell Equations describing the electromagnetic force. The electric force is conservative while the magnetic force is not. 
As we have already seen in our discussion of potential energy, forces and potential energy are closely related. You will see in electricity and magnetism how electrical potential and electrical potential energy will use some techniques from multivariate calculus to develop these concepts more rigorously. We will think about potential energy only for conservative forces because the concept would not fit with non-conservative forces. For this reason, we will see an electric and gravitational potential energy, but we will not, for example, see a magnetic or frictional potential energy. In case you do not remember, a conservative force is a force for which if a body starts at a given point, and goes on any path and returns the same point, the work done by the force is 0. This must be true for all of the points in the plane. By this I mean, if we move an object from point A to point B regardless of how, the same amount of work must be done by the force. It is path-independent. So, in some sense, we can think about each point having a distinct energy associated with it where the work done by a force in moving it from A to B is the difference between the energy at these points. We will discuss this topic in more detail when we cover gravitation later, and we will discuss how we can think of the gravitational force as being the result of a “force field” that permeates all space. To repeat, if the force is conservative, it does not matter which path for determining the work done. We still have a problem though. How do we assign each point an energy value? If we know $8$ J of work are done in taking an object from point A to point B, how do we determine energy values based off of these? Well, what we can do is to assign a value of 0 for the energy at any location. We usually take this location as infinity, which we think of as being infinitely far away from everything we are considering.

AP Physics only covers the gravitational and electrical forces, as such, we will think about potential energy regarding gravitational and electric forces. It is important not to confuse the gravitational potential energy we are talking about here with the gravitational potential energy we were discussing before. Before, we were making an approximation that is only true when objects are close to the earth. However, the gravitational potential energy that we touched on here and that we will fully cover later is more general. To give an introduction, before we assumed that g is constant, but this is not true when objects are far away from the earth. I challenge you when we get to that section to rigorously show that the approximation we made is reasonable.