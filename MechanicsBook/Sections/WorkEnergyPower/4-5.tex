\
\newline
\begin{tikzpicture}
\draw[black] (0,0) circle (0.001cm);
\draw[black] (2,0) -- (10,0);
\draw[black] (2,0) -- (2,6);
\draw[black] (6,1) parabola (10,5);
\draw[black] (6,1) parabola (2.25,5);
\draw[black] (5,-0.5) node[anchor=west] {$Position\left(x\right)$};
\draw[black] (1.5,3) node[anchor=west] {\begin{turn}{90} $Potential \ Energy\left(U\right)$ \end{turn}};
\end{tikzpicture}
\begin{center}
(Figure 4.4.1)
\end{center}
Energy diagrams are ways of showing us where objects will go. We do this by looking at energy diagrams and finding local maximums or minimums. The minimal are the equilibrium points at which the body will stay stable. We can think about this in terms of how we described the relation between forces and potential energy previously. If we have a diagram where we give the potential energy $U$ of an object at various positions $X$, then the slope of the line, the derivative of the function $U\left(x\right)$ will be 0 at a maximum or a minimum, as you should have learned in calculus. However, we know that $\frac{dU}{dx}=F$. So if the slope of the energy curve $$\frac{dU}{dx}$$ is 0, then the force on the object is at 0, and it is in equilibrium. However, we also know that objects will try to lower their energy, so if the ball is at a local minimum, then it will want to stay there, if it is perturbed a little bit from its spot, it will go right back down the minimum. However, this is not the same at a local maximum. If the object is at a local maximum and is perturbed slightly, it will continue going down in energy because the force is bringing it down. Energy diagrams can be helpful for our understanding of chemical reactions. However, in chemical reactions, the bottom side is the reaction coordinate reflecting the extent to which the reaction goes about. So, the reaction starts at medium energy and energy is required to get it over an activation hump, after this, the reaction continues and decreases from this energy. Sometimes, if the reaction is exothermic, the energy will be decreased on the net, but energy is still required for the reaction to go forward. We can parallel this with a ball on a ramp. This energy diagram can be thought of like a ball on a ramp. We know that gravitational potential energy can be thought of as just $mgh$, so the height of the object directly reflects the gravitational potential energy. It should be intuitive where the object would go assuming it starts at the leftmost point. It is not going anywhere. Objects don't just magically go up. Some force needs to provide energy to the objects to move it up. Once the object gets just slightly over the leftmost maximum, the force does not need to be provided for it to get to the minimum. The object will naturally go down. This makes sense. Objects like to lower their potential energy and their energy in general. A lot of chemistry deals with different atoms undergoing reactions that require a temporary raising in energy followed by a lowering in energy. These diagrams are useful to use and understand to understand what energy means for us and to give an intuitive explanation for why objects try to lower their energy. 