We will now begin to deal with work and power. I am sure you have heard about work and power because they are some of the only terms in physics that we can see used in our daily lives. The first thing that you have probably learned about work is that \begin{equation}W=Fd\end{equation} First, here W is work, F is the force, and d is the distance that the object has traveled. First, we need to introduce a new unit called the Joule to begin talking about work. A joule is simply a Newton times a meter. Which is $\frac{kg\cdot m^2}{s^2}$. Of course, it is unclear exactly what we mean by Force and distance here and this definition is not true in general, so it is not particularly useful. A better, but still imperfect definition of work is \begin{equation}W=\vec{F} \cdot \vec{d}\end{equation} Where $\vec{d}$ is the vector representing the distance that the object has traveled. This tells us that the work a force does on an object is the dot product times the position vector of the object. If you know what dot products are, then you should be able to reason when the work done by a force is 0 and when it is at a maximum assuming the force is constant. If you have not yet seen the dot product or do not remember it, I will provide a refresher here. $$\vec{a} \cdot \vec{b}=|\vec{a}| \ |\vec{b}| \ \cos\left(\theta \right)$$ $\theta$ here is the angle between the two vectors. What this means is that if $\vec{F}$ and $\vec{d}$ are along the same direction then $\theta$ is 0 and $$\cos\left(\theta \right)=1$$ so the work is just $$\mid \vec{F}\mid \ \mid\vec{d}\mid$$ $\cos\left(\theta \right)$ is always less or equal to 1 so when $\vec{F}$ and $\vec{d}$ lie along the same direction, work is at a maximum. When $\vec{F}$ and $\vec{d}$ vary by an angle of $90^o$ then $\cos\left(90 \right)=0$ and the work done by the force is 0. So, if the Force is 90 degrees off from the distance of the object, then no work has been done by the force. When theta varies from $0^o$ to $90^o$ degrees, the force progressively declines. You may be wondering, what happens when the angle goes beyond $90^o$. Well when the angle increases from $90^o$ to $180^o$, the work becomes negative. It may be weird to think of negative work, but I think of negative work as occurring when a force is trying hard to make an object switch directions but is unable to. In contrast, positive work is when a force tries to make an object move a certain way, and it is effective. 

A question I always had was, what happens when the Force is changing, and the path is not linear or circular? Well, to do this we need to use calculus and to do it and two and three dimensions, we need to use techniques from multivariate calculus called line integrals. However, we do not need to worry about this level of mathematics. Instead, let us imagine that the force is changing with distance but changes very little when we change the distance($d\vec{x}$). So if we change the distance by a very small $\Delta x$, we can approximate the work as $$\vec{F} \cdot d\vec{x}$$ However, a movement from point a to point b requires a whole bunch of small little changes that can be approximated as $$\vec{F} \cdot d\vec{x}$$ So we can use an integral to sum all of these tiny amounts of work to find the real work. Our approximation becomes more and more accurate as $d\vec{x}$ goes very close to 0. So we write that \begin{equation}W=\int{\vec{F} \cdot d\vec{x}}\end{equation} With this formula, we can do problems where we have a force changing in magnitude with the direction of the force and the path always being the same.

I will not present an example that is not particularly realistic but useful for calculation purposes; we shall imagine a force where \begin{equation}F\left(x\right)=4x^3+x^2+1 \ N\end{equation}, and the force is acting on an object that is moving from $x=2$ to $x=4$. We can assume that $x$ and $dx$ have been normalized with respect to meters. Now we can now just say that work is force times distance because the force is changing, but the directions of the two vectors are the same. So we need to use an integral equation. \begin{equation}W=\int x^3+x^2+1 \ dx \ J= \frac{x^4}{4}+\frac{x^3}{3}+x \ J\end{equation} Plugging in from $x=2$ to $x=4$ we find that   $$\left(64+\frac{64}{3}+4\right)-\left(4+\frac{1}{3}+2\right) J= \frac{242}{3} J$$ This means that the force did about 71 Joules of work in moving the object from $x=2$ to $x=4$. 

Now that we have seen what work is through calculations, we need to get an intuitive understanding of what exactly is going on. We can think about work as being something that forces do to cause objects to move as they want them to. When a force does positive work, it successfully moves an object the way that it wants too. When a force does negative work, it has “failed” at getting the object to move as it “wanted” it too. It is, of course, important to recognize that forces don’t “want” anything but this is a good way of thinking about this topic. 

Now that we have introduced work, we need to think about power. Power is the amount of work that is delivered to something per unit time. In terms of calculus $P=\frac{dW}{dt}$ or $P = \frac{\Delta W}{\Delta T}$. If we assume that the force is not changing and that the force is applied in the same direction as the motion of the object, we can write $$P=\frac{d\left(Fr\right)}{dt}$$ We assumed the force is constant so we can take the F term out and we find that $$P=F\frac{d\left(r\right)}{dt}$$ But $$\frac{dr}{dt}$$ is just the velocity of the object, so we can say that $$P=Fv$$ In the more general case where $\vec{F}$ and $\vec{v}$ are not necessarily in the same direction, we have that $P=\vec{F}\cdot \vec{v}$. It can be confusing what exactly the power is here. For clarification, the power we are talking when we write \begin{equation}P=\vec{F} \cdot \vec{v}\end{equation} is the power delivered to the object by force. So, if for example, an object is moving horizontally across the ground on earth. Gravity is not providing any power because the force and the velocity are perpendiculars, so the dot product goes to zero. The intuition for power is pretty simple. For example, imagine a person trying to move an object from point A to point B. They perform the task but the object experiences a lot of friction, and it takes a time $T_1$. Now imagine an alternate reality where the person asks his/her friend to help, and the two of them move the object from point A to point B in time $T_2<T_1$. Because $T_2<T_1$ the two people move the object with more power(together) because the work of moving the object across the floor is constant, but the time has decreased. 