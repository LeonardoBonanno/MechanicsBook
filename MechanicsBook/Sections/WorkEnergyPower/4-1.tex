\begin{tikzpicture}
\draw[black] (0,0) -- (0,8);
\draw[black] (0,0) -- (12,0);
\draw[black] (0,8) .. controls (2,2) .. (12,8);
\draw[black] (4,-0.5) node[anchor=west] {$Possible \ Energy \ States$};
\draw[black] (-0.75,3.25) node[anchor=west] {\begin{turn}{90} $Potential \ Energy$ \end{turn}};
\end{tikzpicture}
\newline

We will now begin discussing energy, power, and work after having talked about Newton’s laws and dynamics. These will serve as an alternative to Newton’s laws that we can use to analyze certain situations. When I first began learning about energy, I was plagued by one question: “What the hell is energy.” The description I was given was “the ability to do work.” This definition is correct, but because AP Physics does not deal much with thermodynamics, it is not particularly useful.  I mean, we hardly use energy to think about objects doing work in AP Physics. Instead, we think about energy in terms of forces doing work on objects. Primarily because we will often deal with energies that are negative and in that case, what does having a negative ability to do work even mean. In fact, in chemistry, almost all of the energies that one deals with are negative. So, how should you think about energy in this course? Well, I would recommend thinking about energies in the same way we think about energy in chemistry. Energy is a way of representing how much a system does not want to be the way it is, and subsequently how much work it took for the system to get to the way it is. This is not a technical definition, and we will see the technical definition later but I want you to think about this now, and what types of things would signify a high energy. A few examples would be, a pencil balancing on its pointed end, two positive point charges sitting very close to each other, and a ping-pong ball floating high in our sky. When you think about these examples, one thing should become clear to you; the objects will not stay in these positions. The objects will naturally try to lower their energy. This is why I like thinking about energy as a bad thing, and this description will almost always work for our purposes.  