\vspace{5mm}
\begin{tikzpicture}
\draw[circle] (0,0) circle (0.0001cm);
\draw[->] (4,0) -- (8,2);
\draw[black] (6,-0.5) node[anchor = west] {$V_x$};
\draw[black] (6,-0.5) node[anchor = west] {$V_x$};
\draw[dashed] (4,0) -- (8,0);
\draw[dashed] (8,0) -- (8,2);
\draw[black] (8.125,1) node[anchor = west] {$V_y$};
\draw[black] (5.75,1.5) node[anchor = west] {\begin{turn}{45} $\vec{V}$ \end{turn}};
\end{tikzpicture}
\newline

The first topic that you learn about when you begin studying physics is kinematics. Velocity is simply the derivative of position with respect to time. Quite frankly, I do not like the definition being given in this way. Velocity is a measure of how the position of something is changing over time. Normally, this something is just an object, but it does not necessarily have to be. For example, the center of mass of a system can have a velocity as well. We will see this later. As a reminder, our precise formula for velocity is $ \vec{v}= \frac{d\vec{r}}{dt}$. Where $\vec{r}$ is the position of our "thing." We assume that $\vec{r}$ is a position vector. We can normally assume that if a problem deals with the position of something changing over time, the original position of the thing should be defined as the origin. This is usually very convenient. Velocity, after position, is the second physical quantity that is seen as a vector in the standard physics curriculum. It is important to understand that we can view velocity in general as a vector of the form \begin{equation} \vec{v}= <x,y,z>\end{equation} Where $|\vec{v}|$ is the magnitude of the velocity and \begin{equation}\frac{\vec{v}}{|\vec{v}|}\end{equation} is the direction of the velocity. In some sense, we can think of the components of the velocity as not impacting each other. We can think of them as being in some sense independent. By this I mean, if a force is imparted onto an object that changes its velocity. The change in velocity in each direction is only impacted by force imparted in that direction. This will prove very relevant when we begin studying free fall and the throwing of objects. Now that we have defined velocity, we can define speed. Speed is simply the magnitude of the velocity vector. If you are given the speed of an object and direction of this speed, you can find the velocity vector of the object. This is simple enough when the speed is constant; however, in practice, the speed is changing, so if we want to think about the actual path of the object, it is usually easier to work with velocity than with speed. In fact, velocity is used far more often than speed and speed is the only common scalar that we will encounter in kinematics. Once we start studying acceleration and jerk, we will see that the magnitudes of these vectors, unlike speed, are simply called “the magnitude of XXX." 
\clearpage
